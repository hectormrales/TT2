% !TeX root = main.tex

%=========================================================
\chapter{Introducción}
\label{cap:introduccion}

Este documento contiene el análisis de requerimientos del proyecto ``{\em Sistema Inteligente para la Identificación y Seguimiento de Derechos de Niñas, Niños y Adolescentes (NNA) afectados por feminicidios en México}'' que servirá como base para el análisis, diseño, construcción, pruebas y aceptación del proyecto.

El documento fue elaborado por Herrera Ramírez Emilio Alejandro y Morales Martínez Héctor Alberto, estudiantes de Ingeniería en Sistemas Computacionales de la Escuela Superior de Cómputo del Instituto Politécnico Nacional, bajo la dirección del M. en C. Ulises Vélez Saldaña y el Dr. Gabriel Hurtado Aviles, durante el periodo académico 2025-2026 en la Ciudad de México.

%---------------------------------------------------------
\section{Presentación}
\label{sec:presentacion}

Entre 2010 y 2023, más de 8,000 mujeres fueron víctimas de feminicidio en México según la Secretaría de Gobernación. Detrás de estas cifras existen miles de niños, niñas y adolescentes que quedaron huérfanos. La Red por los Derechos de la Infancia en México (REDIM) estima al menos 3,500 NNA que perdieron a su madre por feminicidio, pero no existe un censo oficial ni sistema de seguimiento estructurado. La información permanece dispersa en notas periodísticas, comunicados de fiscalías estatales y archivos de organizaciones civiles, lo que impide dimensionar el problema, identificar patrones geográficos o temporales, y diseñar políticas públicas efectivas para garantizar los derechos de estas infancias afectadas.

Las organizaciones de la sociedad civil, como la Fundación Futuro con Derechos, dedican recursos considerables a la revisión manual de medios de comunicación buscando casos donde se mencione a hijos de víctimas. Sin embargo, la metodología manual es insuficiente ante el volumen de información disponible, resultando en cobertura limitada y muchos casos que pasan desapercibidos. Se requiere un sistema automatizado que recolecte, procese y estructure esta información dispersa para transformarla en datos consultables y accionables.

El propósito de este documento es presentar el análisis detallado de requerimientos funcionales y no funcionales del sistema, describir la arquitectura técnica implementada mediante prototipos incrementales, documentar las estrategias de recolección de datos y procesamiento de lenguaje natural, y establecer las bases para las fases siguientes del proyecto. Este documento está dirigido a los directores del trabajo terminal, sinodales evaluadores, organizaciones de la sociedad civil interesadas en la problemática, y desarrolladores que participen en la evolución del sistema.

El documento debe utilizarse como referencia técnica para comprender las decisiones de diseño, validar el cumplimiento de objetivos establecidos en TT1, identificar limitaciones reconocidas que serán abordadas en TT2, y servir como punto de partida para la documentación de usuario final y manuales técnicos que se desarrollarán en la segunda fase del proyecto.

%---------------------------------------------------------
\section{Organización del contenido}
\label{sec:organizacion}

El presente documento se estructura en ocho capítulos que describen de manera integral el desarrollo del sistema:

El \textbf{Capítulo 1: Introducción} presenta el contexto del proyecto, planteamiento del problema, justificación social y tecnológica, objetivos generales y específicos, así como los alcances y limitaciones de la fase TT1.

El \textbf{Capítulo 2: Marco Teórico} establece los fundamentos conceptuales necesarios para comprender el sistema, incluyendo procesamiento de lenguaje natural, técnicas de web scraping, clustering, modelado de tópicos, métricas de evaluación y tecnologías utilizadas (Python, Flask, Docker, scikit-learn).

El \textbf{Capítulo 3: Estado del Arte} analiza proyectos relacionados a nivel nacional e internacional, incluyendo el Mapa de Feminicidios en México, Data Cívica, sistemas de monitoreo de medios, y plataformas académicas de análisis de noticias, identificando fortalezas y diferencias con el presente proyecto.

El \textbf{Capítulo 4: Metodología de Desarrollo} documenta la evolución a través de tres prototipos: P0 (web scraping directo fallido), P1 (sistema básico con RSS y K-Means), y P2 (sistema avanzado con triple estrategia de recolección y detector especializado), describiendo implementación, resultados y lecciones aprendidas de cada fase.

El \textbf{Capítulo 5: Desarrollo y Arquitectura de la Solución} detalla la arquitectura Docker implementada, componentes del sistema (recolector multi-fuente, detector dual-etapa, pipeline PLN, API REST, dashboard web), flujo de datos, y decisiones técnicas de diseño.

El \textbf{Capítulo 6: Resultados Preliminares y Pruebas} presenta las métricas de desempeño del Prototipo 2, incluyendo cobertura de recolección (281 noticias únicas), precisión del detector (90\% con 10\% falsos positivos), análisis de clusters generados, y evaluación de calidad mediante coherencia de tópicos LDA.

El \textbf{Capítulo 7: Conclusiones Parciales} sintetiza los logros de TT1, valida la viabilidad técnica del enfoque propuesto, reconoce limitaciones identificadas, y establece las bases para la continuación del proyecto.

El \textbf{Capítulo 8: Trabajo a Futuro} define el alcance de TT2, priorizando la implementación de clasificación semántica con transformers, extracción de entidades nombradas (NER), migración a base de datos relacional, ampliación de cobertura temporal, y desarrollo de interfaz de usuario completa para organizaciones civiles.

%---------------------------------------------------------
\section{Notación, símbolos y convenciones utilizadas}
\label{sec:notacion}

El documento emplea las siguientes convenciones de notación y estándares de documentación:

\begin{itemize}
    \item \textbf{Nomenclatura de Prototipos:} Los prototipos se identifican como P0, P1 y P2, representando las fases de desarrollo Prototipo 0 (fallido), Prototipo 1 (básico) y Prototipo 2 (avanzado), respectivamente.
    
    \item \textbf{Identificadores de Componentes:} Los componentes del sistema se nombran en formato \texttt{snake\_case} según convenciones Python (ejemplo: \texttt{data\_collector.py}, \texttt{simplified\_analyzer.py}).
    
    \item \textbf{Acrónimos y Abreviaturas:}
    \begin{itemize}
        \item \textbf{NNA:} Niñas, Niños y Adolescentes
        \item \textbf{REDIM:} Red por los Derechos de la Infancia en México
        \item \textbf{PLN:} Procesamiento de Lenguaje Natural
        \item \textbf{RSS:} Really Simple Syndication
        \item \textbf{API:} Application Programming Interface
        \item \textbf{REST:} Representational State Transfer
        \item \textbf{TF-IDF:} Term Frequency - Inverse Document Frequency
        \item \textbf{LDA:} Latent Dirichlet Allocation
        \item \textbf{DBSCAN:} Density-Based Spatial Clustering of Applications with Noise
        \item \textbf{NER:} Named Entity Recognition
        \item \textbf{TT1/TT2:} Trabajo Terminal semestre 1 y semestre 2
    \end{itemize}
    
    \item \textbf{Referencias Cruzadas:} Las referencias a figuras, tablas y capítulos utilizan comandos LaTeX estándar (\verb|\ref{}|, \verb|\label{}|) para mantener consistencia automática.
    
    \item \textbf{Código Fuente:} Los fragmentos de código se presentan utilizando el entorno \texttt{lstlisting} con resaltado de sintaxis para Python, con numeración de líneas cuando sea relevante.
    
    \item \textbf{Métricas y Valores:} Las métricas de evaluación se expresan en porcentaje cuando representan tasas (precisión, recall), y en valores absolutos para conteos (número de noticias, features, clusters).
    
    \item \textbf{Énfasis Tipográfico:} 
    \begin{itemize}
        \item \textit{Cursivas} para términos técnicos en su primera aparición o conceptos destacados
        \item \textbf{Negritas} para elementos de especial relevancia o términos clave en definiciones
        \item \texttt{Monoespaciado} para nombres de archivos, comandos, código, URLs y parámetros técnicos
    \end{itemize}
    
    \item \textbf{Estándar de Documentación:} El documento sigue las convenciones de trabajos terminales de ESCOM-IPN, con estructura de capítulos, formato de referencias bibliográficas estilo APA, y numeración jerárquica de secciones.
    
    \item \textbf{Notación Matemática:} Las fórmulas matemáticas (similitud coseno, TF-IDF, métricas de evaluación) se expresan utilizando el entorno matemático de LaTeX con la notación estándar de álgebra lineal y estadística.
\end{itemize}

El documento utiliza el sistema de gestión de referencias bibliográficas BibTeX con estilo \texttt{apalike}, asegurando consistencia en las citas y formato de la bibliografía conforme a estándares académicos internacionales.