% Resumen Ejecutivo del Proyecto

\section*{Contexto del Proyecto}

Entre 2010 y 2023, más de 8,000 mujeres fueron víctimas de feminicidio en México, dejando miles de niñas, niños y adolescentes en situación de orfandad. La información sobre estos casos permanece dispersa en notas periodísticas, comunicados de fiscalías y archivos de organizaciones civiles, impidiendo dimensionar el problema de manera estructurada y diseñar políticas públicas efectivas.

\section*{Objetivo General}

Desarrollar un sistema inteligente que automatice la recolección, procesamiento y estructuración de información relacionada con niñas, niños y adolescentes afectados por feminicidios en México, mediante técnicas de procesamiento de lenguaje natural y análisis de datos, para transformar información dispersa en datos consultables y accionables.

\section*{Alcance de la Fase TT1}

Durante el primer semestre del Trabajo Terminal se desarrollaron e implementaron tres prototipos incrementales:

\begin{itemize}
    \item \textbf{Prototipo 0 (P0):} Aproximación inicial mediante web scraping directo que identificó limitaciones técnicas y legales de acceso a fuentes periodísticas.
    
    \item \textbf{Prototipo 1 (P1):} Sistema básico implementando recolección mediante RSS feeds, vectorización TF-IDF y clustering con K-Means para agrupar noticias relacionadas.
    
    \item \textbf{Prototipo 2 (P2):} Sistema avanzado con arquitectura Docker, triple estrategia de recolección de datos (RSS, scraping de listados, APIs periodísticas), detector de relevancia de dos etapas, pipeline completo de PLN, API REST y dashboard web para visualización.
\end{itemize}

\section*{Resultados Principales}

El Prototipo 2 demostró la viabilidad técnica del enfoque propuesto con los siguientes resultados:

\begin{itemize}
    \item Recolección automática de 281 noticias únicas en periodo de prueba
    \item Precisión del 90\% en detección de casos relevantes
    \item Tasa de falsos positivos del 10\%
    \item Cobertura de múltiples entidades federativas
    \item Generación automática de clusters temáticos coherentes
\end{itemize}

\section*{Trabajo a Futuro (TT2)}

La segunda fase del proyecto se enfocará en:

\begin{enumerate}
    \item Implementación de clasificación semántica con transformers (BERT multilingüe)
    \item Extracción de entidades nombradas (NER) para identificación automática de nombres, ubicaciones y fechas
    \item Migración a base de datos relacional (PostgreSQL)
    \item Ampliación de cobertura temporal (2015-2025)
    \item Desarrollo de interfaz de usuario completa para organizaciones civiles
    \item Sistema de validación y corrección manual supervisada
    \item Módulo de análisis estadístico y visualizaciones avanzadas
\end{enumerate}

\section*{Impacto Esperado}

El sistema permitirá a organizaciones como la Fundación Futuro con Derechos y la REDIM contar con información estructurada y actualizada sobre NNA afectados por feminicidios, facilitando el diseño de estrategias de acompañamiento, defensa de derechos y generación de políticas públicas basadas en evidencia.
