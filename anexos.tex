% Anexos del Documento

\section{Glosario de Términos}

\subsection*{NNA}
Niñas, Niños y Adolescentes. Término utilizado en el marco jurídico mexicano para referirse a personas menores de 18 años.

\subsection*{Feminicidio}
Delito tipificado en el Código Penal Federal como el asesinato de una mujer por razones de género, con agravantes específicas contempladas en el artículo 325.

\subsection*{Procesamiento de Lenguaje Natural (PLN)}
Rama de la inteligencia artificial que permite a las computadoras entender, interpretar y generar lenguaje humano de manera útil.

\subsection*{Web Scraping}
Técnica de extracción automática de información de sitios web mediante programas informáticos.

\subsection*{RSS (Really Simple Syndication)}
Formato XML utilizado para distribuir contenido actualizado de sitios web de forma automatizada.

\subsection*{TF-IDF}
Term Frequency - Inverse Document Frequency. Medida estadística que evalúa la importancia de una palabra en un documento dentro de una colección.

\subsection*{Clustering}
Técnica de aprendizaje no supervisado que agrupa elementos similares en conjuntos denominados clusters.

\subsection*{API REST}
Application Programming Interface basada en el estilo arquitectural REST que permite la comunicación entre sistemas mediante protocolo HTTP.

\section{Configuración del Entorno de Desarrollo}

\subsection*{Requisitos del Sistema}
\begin{itemize}
    \item Python 3.9 o superior
    \item Docker Desktop 4.0 o superior
    \item 8 GB RAM mínimo (16 GB recomendado)
    \item 10 GB espacio en disco
    \item Sistema operativo: Windows 10/11, macOS 11+, o Linux (Ubuntu 20.04+)
\end{itemize}

\subsection*{Librerías Python Principales}
\begin{itemize}
    \item \texttt{scikit-learn} 1.3.0 - Algoritmos de machine learning
    \item \texttt{Flask} 2.3.0 - Framework web
    \item \texttt{pandas} 2.0.0 - Análisis de datos
    \item \texttt{beautifulsoup4} 4.12.0 - Parsing HTML
    \item \texttt{feedparser} 6.0.10 - Procesamiento RSS
    \item \texttt{spacy} 3.6.0 - Procesamiento de lenguaje natural
\end{itemize}

\section{Ejemplos de Casos de Uso}

\subsection*{Caso de Éxito: Detección de Cluster de Feminicidios en Chihuahua}

Durante las pruebas del Prototipo 2, el sistema identificó automáticamente un cluster de 12 noticias relacionadas con casos de feminicidio en el estado de Chihuahua durante un periodo de 3 semanas, permitiendo detectar un patrón geográfico-temporal que no había sido documentado previamente por las organizaciones civiles.

\subsection*{Caso de Refinamiento: Reducción de Falsos Positivos}

La implementación del detector de dos etapas redujo la tasa de falsos positivos del 28\% (P1) al 10\% (P2), mejorando significativamente la calidad de los datos recolectados y reduciendo el tiempo de validación manual requerido.

\section{Código de Ejemplo: Detector de Relevancia}

\begin{lstlisting}[language=Python, caption=Fragmento del detector de relevancia de dos etapas]
def is_relevant(title, description):
    """Detector de dos etapas para identificar noticias relevantes"""
    
    # Etapa 1: Filtro rapido con palabras clave
    keywords = ['feminicidio', 'asesinato de mujer', 'hijos',
                'huerfanos', 'menores', 'ninos']
    text = (title + ' ' + description).lower()
    
    if not any(keyword in text for keyword in keywords):
        return False
    
    # Etapa 2: Validacion semantica con TF-IDF
    vector = vectorizer.transform([text])
    similarity = cosine_similarity(vector, reference_vectors)
    
    return similarity.max() > 0.3
\end{lstlisting}

\section{Diagramas Complementarios}

\subsection*{Flujo de Procesamiento de Noticias}

El flujo completo de procesamiento incluye las siguientes etapas:
\begin{enumerate}
    \item Recolección desde múltiples fuentes
    \item Deduplicación por URL y similitud de contenido
    \item Detección de relevancia (dos etapas)
    \item Extracción de características con TF-IDF
    \item Clustering automático
    \item Almacenamiento en formato JSON
    \item Exposición mediante API REST
    \item Visualización en dashboard web
\end{enumerate}

\section{Referencias de Contacto}

\subsection*{Organizaciones Colaboradoras}
\begin{itemize}
    \item \textbf{Fundación Futuro con Derechos} \\
          Organización enfocada en la defensa de derechos de NNA en situación de orfandad por feminicidio
    
    \item \textbf{Red por los Derechos de la Infancia en México (REDIM)} \\
          Red de organizaciones civiles dedicadas a la promoción y defensa de los derechos de la infancia
\end{itemize}

\subsection*{Autores del Proyecto}
\begin{itemize}
    \item \textbf{Herrera Ramírez Emilio Alejandro} \\
          Estudiante de Ingeniería en Sistemas Computacionales - ESCOM IPN
    
    \item \textbf{Morales Martínez Héctor Alberto} \\
          Estudiante de Ingeniería en Sistemas Computacionales - ESCOM IPN
\end{itemize}
